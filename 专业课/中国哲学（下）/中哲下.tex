\documentclass{article}
\usepackage[UTF8]{ctex}
\usepackage{tabularx}
\usepackage{graphicx}
\usepackage{graphics}
\usepackage{amsmath}
\usepackage{amsthm}
\usepackage{amssymb}
\usepackage{geometry}

\title{\heiti 中国哲学(下)复习纲要\footnote{本文档整理自《宋明理学十五讲》,属无偿分享,禁止任何形式的交易行为。}}
\author{Sulley}
\date{}

\begin{document}
\maketitle
\tableofcontents
\newpage

\section{第一讲、韩愈与儒学复兴运动}
\subsection{一、	中晚唐的文化氛围}
隋朝科举制度的建立在一定程度上打破了魏晋六朝门阀士族的垄断,但是它没能在真正意义上普及,在李唐一代的政治斗争中,很多的世家大族和新进的进士集团之间的斗争。唐代士大夫阶层的精神根底,主要是道教和佛教,导致在这个时期真正具有儒家正统观念的人越来越少,这正是儒学复兴运动产生的背景。
儒学复兴有两方面的意义:一是在佛道的兴盛中,如何保持中国历代延承的儒家生活方式;二是如何应对在政俗两方面产生深刻影响的宗教迷狂,倡导一种理性主义的精神态度。
\subsection{二、	古文运动和儒学复兴运动}
中晚唐的儒学复兴运动影响深远,使得士大夫的基本精神风格发生变化,即理性化的特征和普遍怀疑的态度。
\subsection{三、	韩愈的贡献}
一是盛倡华夷之辨。韩愈倡导华夷之辨本身是中华固有文化的又一次自觉,是中华固有文化对自己文化、文明主体性的又一次自觉。
二是排挤佛教。
三是发明道统。这是为自己的儒家传承谱系确立一个合法性、正当性的基础。
四是古文运动。这就是要找到一种适合传达思想的文风。主要是文体上的突破,重建一种适合表达思想的文风。
五是表彰《大学》。
六是奖掖后学。儒家一直重教化、师道。
\subsection{四、	韩愈的思想}
韩愈在《原道》中对仁义道德这四个字给出了明确的定义。他说“博爱之谓仁,行而宜之之谓义,由是而之焉之谓道,足乎己无待于外之谓德”。它重新发现了儒学的基本精神。“行”的是“博爱”,所以仁义不分开,这就防止了博爱沦为兼爱。然后沿着仁义这条路走才是道。“仁与义为定名,道与德为虚位”真正为道统赋予内容。儒家是一种理性的生活态度、合道理的生活方式、符合人的本质的生活方式。
韩愈还有他的人性论,是从现实的人性这个角度讲的。他认为,生而具有的、不需要学习的那个部分就是人性。因此人性有上中下三品,恶的可以引导为善的。
韩愈没能找到儒学复兴的道路吗,只是模糊地找到了一个方向,但是具体问题没有拈出。

\section{第二讲、北宋士大夫精神与宋初三先生}
\subsection{一、	北宋的开国规模和士大夫精神的觉醒}
北宋的思想艺术成就、文学高度是空前绝后的。宋仁宗有“畏”,一方面带来宽容——养士,另一方面带来疑忌——严格的管制——权力制衡。北宋虽然富庶,但整个国家一直是岌岌可危的,这种盛世隐忧的局面构成了士大夫精神觉醒的基本要素,再加上北宋有养士之风,北宋一直强调士大夫精神,鼓励士大夫人格的提升,某种以天下为己任的精神也就自觉地产生了。
首先,士大夫精神有两种倾向:政治改革诉求和对整个文化的焦虑、对人伦秩序的焦虑,由此产生对人伦秩序安排及其背后的哲学理由的关切。
第二是困穷苦学的普遍性。
第三是道德的自我节制力。北宋士大夫对于自己的私人生活有一种近乎宗教的虔诚。
第四是排抵佛老,倡导儒家的理性生活成为一种普遍的人生态度。
\subsection{二、	胡瑷的著作与思想}
宋初三先生就是胡瑷、孙复和石介。
胡瑷是北宋早期士大夫精神的代表,幼时十分刻苦。胡瑷最著名的著作是《周易口义》,是当时影响最大的《周易》注解之一。
第一,《周易口义》的解《易》体例。他是在王弼《注》和孔颖达《疏》的基础上的进一步发展,有三点值得强调。第一,胡瑷强调《序卦传》。这也就意味着卦与卦之间有相互转化的关系,不是随意的。儒家强调的是不同处境之间有转化的可能。第二,胡瑷对《注》《疏》的调整和突破更多地出现在讲“成卦之义”的场合,“成卦之义”就是讲某一卦象为什么是这个卦象。胡瑷强调二体义,即通过上下卦之间的关系来讲成卦之义。第三,对《乾》之四德的强调。所谓四德就是“元亨利贞”。
第二,《周易口义》的思想。总体上没有超出汉唐元气论的格局,但强调了“生生之德”,这是巨大的突破。王弼说“天地以本为心”、“以无为本”,陷入了虚无主义。胡瑷则说“天地以生成为心”,所以天地就是永恒的绝对创造,就不会有虚无的阶段。
\subsection{三、	孙复与石介}
石介最著名的文章是《怪说》,把文章、佛、老称为“三怪”。
在经学上,孙复比石介更重要,主要体现在《春秋》学上,著作为《春秋尊王发微》。华夷之辨为北宋文化自觉奠定了牢固的基础。
北宋士大夫精神人格的普遍特点就是不苟且,所有的地方都是严谨的,需要一种高度的自律。

\section{第三讲、周敦颐的哲学}
\subsection{一、	太极与诚}
周敦颐的核心是太极。《太极图说》的第一句话是“无极而太极”,《通书》的一个重要观念是诚。“诚”和“太极”都有本体论的意义,是万物本体和万物本根的意思。
“无极而太极”似乎是说无极在太极之上。那么,什么是“无极”?按照朱子的解释,就是无形的意思,这与“无极之真,二五之精,妙合而凝”,在这里“太极”被称为“真”。“二”是阴阳,“五”是五行,阴阳五行的精妙合而凝结成万物。“太极”强调的是“真”和“实有”。“无极而太极”讲的是无形的至真存在,就是始终生生不已的至真的存在,是一种必然的创生,是永无止境的。“太极”就是“诚”,也就是“真”。世界是真实的、真诚的,不用怀疑。世界是一个生生不息的至真存在。
接下来是“太极动而生阳”。太极既然是无形的,怎么能动呢?先放下。接下来是“动极而静,静而生阴,阴极复动”。在太极图里,中间的小圈就是“太极”,也就是说已经分化为阴阳之后,太极仍然在阴阳之中,以更丰富的形态表现了出来。那么“极”怎么理解呢?大概是动静互为条件,从而阴阳互为条件。
接下来就是五行。“阳动阴静而生水火”。接下来才生木金土。但是五行的运转不是这样的,二是木火金水土,土在中间,这也对应了仁义礼智信五常。五行土下面的小圈也是“太极”。从时间上看,“五气顺布,四时行焉”,时间的观念才真正展开。在《通书》里又从“诚”这个概念来阐发。周敦颐把整个世界的变化分成了两个部分,又把“诚”和“元亨利贞”四德放在一起,“元亨,诚之通也,利贞,诚之复也”。诚之通和诚之复构成宇宙运化的两个阶段。诚之通是创生的过程,诚之复是完成的过程。
现在来总结一下:首先是一个至真的、生生不已的存在,至真的无形的存在。这种至真的无形的存在必然体现为阴阳的综合体、动静的综合体。阴阳动静的综合体经过复杂的凝合过程展现出来五行,其中都是有太极的。
\subsection{二、	人与万物}
五行以下就是万物化生,就有了人与万物。二气五行共同化生万物。周敦颐强调人在万物中的独特地位。《周易口义》认为人和天地的不同在于人有忧,这是人的优势。周敦颐说的人的特殊性不同,周敦颐强调的是“得其秀而最灵”,在于人最完整地禀得了天地中最精华的东西,是天地的精华最直接的体现。胡瑷在某种意义上强调的还是天人的差异,而周敦颐则体现出了天人合一,即人的规律就是自然规律的集中体现。天地的本性就是人的本性,人类的所有道德法则都源于天地的本性。儒家把人类的道德价值跟天地的本性关联起来,而这种关联不是源自主观的构造,而是有哲学洞见。
那么最灵的人为什么需要治理呢?这只能结合善恶了。
\subsection{三、	圣人}
首先,谁来治理社会呢?答案是圣人。圣人能够发现人类社会的价值原理,发现人类社会价值的根据所在。那么什么是圣人?周敦颐用三个概念来解释,即诚、神、几。“寂然不动,诚也”,是没有主动欲求的意思。“感而遂通者,神也”,遇到有物来感发就能有所通达。不是主动的欲求,而是为物所感。但什么叫主动的欲求?欲是分外的、多出来的那部分的追求。一个人要依本分而行,不做非分之想,则无时不寂,无时不感。“动而未形,有无之间者,几也”要联系“诚无为,几善恶”。诚是没有任何主动的作为的,也就是不能逾越自己的分内而为。那么“几”就是“动而未形,有无之间”,是对寂和感、诚和神关系的补充。正是“几”这种“动而未形,有无之间”的主动状态才能够把寂和感这两者真正关联起来,才能使得“寂然不动、感而遂通”在现实生活中成为可能。
“动而未形,有无之间”就是本分内的追求,不去刻意地、额外地追求。“诚、神、几曰圣人”,圣人不能有过分的主动欲求,一旦有过分的主动欲求,人的神志就昏了,有过分的主动欲求,人就有了私心。周敦颐强调“公则明”。正因为圣人有公、明,才能体察世界的本性,能够发现人类价值的基本准则。
\subsection{四、	教化与治理}
有了圣人,就要有具体的治理方法和措施。
首先是师道。周敦颐说人有“刚善、刚恶、柔善、柔恶”。因为太极有阴阳,阴阳体现为刚柔。刚柔要得其中。这就需要师道。
然后是纯心,强调的是君主的修身。所有的行为都要不违背仁义礼智信。
其次是礼乐。周敦颐说“礼,理也”。实际上是在为儒家的“礼”找到一个背后的根据。“乐者,和也”。第一,乐以正为本。好的乐一定来自于好的治理。第二,好的乐是“淡而不伤,和而不淫”的。
最后是用刑。周敦颐强调用刑来制止人过度的欲望。
\subsection{五、	志学}
学要立志,立志的标准是“圣希天”,即圣人追求的是天的境界;“贤希圣”;“十希贤”。周敦颐提出了儒家士大夫的理想和目标“志伊尹之所志”;“学颜子之所学”。那么怎么成为圣人呢?“圣可学。一为要”,就是要专一。“一者,无欲也”,就不是过度的欲望。

\section{第四讲、《易》兼体用:邵雍的思考}
\subsection{一、	观物}
邵雍的整个人生态度建立在静观明理上。观物的态度是一种非常客观冷静的态度,他要把一切主观的人的情感的因素都清除掉。
“观之以理”是说如实地、客观地来面对事情。因为人的主观意识使我们不能冷静地客观地来看待事物。“以物观物,性也;以我观物,情也。性公而明,情偏而暗”。在邵雍看来,性、命、理之间是有共同性的,在我们的身上就是性,在物的身上就是理,到了“以物观物”就是性,“以我观物”就是情。“以道观道,以性观性,以心观心,以身观身,以物观物”,这样才能两不伤。
\subsection{二、	体用}
邵雍在讲“用”的时候有很多的复杂的讲法。在说《周易》的时候,他说“用以体为基,故存一也;体以用为本,故去四也”。体是从静的方面来说,用则从动来说。用以体为基础,体以用为目标。
关于体用关系,邵雍还说“天主用,地主体。圣人主用,百姓主体,故‘日用而不知’”。这还是讲动静的关系。邵雍认为,体主静,用主动,体对应阴、地、方,用对应阳、天、圆。表面上看,体更根本,但实际上,用更重要。
\subsection{三、	体以四立}
这个“四”是什么意思呢?邵雍说“物之大者,无若天地,然而亦有所尽也。天之大,阴阳尽之矣;地之大,刚柔尽之矣。阴阳尽而四时成焉,刚柔尽而四维成焉”。第一句说,天地是最大的,但也有尽头。第二句出现了阴阳刚柔。第三句,阴阳是时间的基础,刚柔是空间的基础。阴阳刚柔出现之后,接着继续往下分。阴阳就分成了太阳、少阳、太阴、少阴,这已经成四了,对应日月星辰。刚柔也分为太刚、少刚、太柔、少柔,对应水火土石。日月星辰对应暑寒昼夜,水火土石对应雨风露雷。接着暑寒昼夜变物之性情形体;雨风露雷化物之走飞草木。之后,万物就生出来了。万物之生没有一个由低到高或者由高到低的秩序,谁也不比谁优越,是复杂的感应变化的过程。到了走飞草木这里,接下来是人。为什么人这么重要,因为“暑寒昼夜无不变,雨风露雷无不化,性情形体无不感,走飞草木无不应”,但是贯通在这些变化之中的,所以“灵于万物”。由于人可以贯通,所以人有主动性。
下面来看第二部分。这部分讲一个价值秩序,皇帝王伯。“夫昊天之尽物,圣人之尽民”。昊天有四府,春夏秋冬,阴阳升降于其间;圣人也有四府,《易》《书》《诗》《春秋》,《礼》《乐》升降于其间。春夏秋冬意味着万物收藏,《易》《书》《诗》《春秋》对应了生长收藏的秩序。昊天强调的是时,是不可逆转的必然性,凡物必有春夏秋冬。圣人讲的是经。经是以经法天。这时人的主体性、主观努力已经出来了。到此,他又乘四。生长收藏从生生、生长、生收、生藏到藏生、藏长、藏收、藏藏。从生生到生藏是一个递降顺序,对应了皇帝王伯,是阴阳消长。藏生到藏藏是秦晋齐楚。生生这个阶段是万物最蓬勃的阶段,藏藏是最寂寥的阶段。这就是一个治理之道,是一个递降的过程。
第三部分也是四分的,就变到元会运世。日月星辰是一个光芒的等级,构成了复杂变化,也就是阴阳消长的变化。“日经天之元,月经天之会,星经天之运,辰经天之世”。接下来又乘四。于是有了元元到世世的顺序。皇帝王伯不是指治理时间的长短,而是指治理效果的差别和治理原则的适用范围大小。
第四部分就是士农工商,讲的是治理的效果。邵雍又乘四。从士士到商商。《观物内篇》强调静、体,人的主观能动性是有限的,从而要顺应时势。
\subsection{四、	用因三尽}
“三”把圆引入了。(这一段逻辑太无语了,总之就是在强调人的主观性,是在天地万物客观之理的深刻洞察的前提下。)

\section{第五讲、自立吾理:程颢哲学的精神(上)}
\subsection{一、	对佛教的批判}
程颢的贡献在于,他在批判佛教的同时树立起了儒学的基本方向,他说“自明吾理”,这是儒学的一个明确的号召。就是要为儒家生活方式找到形而上学基础,奠定哲学基础。
程颢的第一个批判是“以生死恐动人”。但是活着的事情都没明白,去纠结死的事情干嘛呢。第二,佛教追求解脱。程颢说,佛教一开始是一个自利的心。为什么有,因为没有看到万物一体,二是从自己的身体出发,要去根尘。佛教说要出家,不用承担社会义务,不承担人类繁衍的义务了。程颢认为要远离佛教。
\subsection{二、	道学话语的建构}
第一,程颢建立起了一种衡量各种思想正确与否的判断标准。第一条是普遍性原则,要适用于万物。第二条是“一本”原则。本就是根,二本就是不同的根。程颢认为一定是从一个核心的原理出发,能够从这个原理依次生发出来。
第二,是对道学基本概念的贡献。比如“天理”。天理二字的确立,真正为儒家生活奠定了哲学的基础,就是要证明儒家生活方式的合理性。
第三,对儒家的根本价值“仁”的深入阐发。这有三个方面:以知觉论仁,以一体论仁,以生意论仁。
第四,他强调形上形下的区别。这为道学思想的发展,开辟出了一个巨大的思考空间,从而引入了道与器,理与气之间的关系等问题。
最后,在修身的基本原则上,程颢强调“敬”。
\subsection{三、	天理}
天理是什么?第一点,天理是普遍的。第二点,天理是客观的。天理既然是客观的,那么“小我”又怎么会参与其中?天理有其自然性,非人为的,自然而然的,顺着自然而然去做就是你应该做的。所以对于善的,要表彰,对恶的,要厌恶。一切都是自然的。天理出自对客观物理的考察。程颢认为善恶都是天理,但是人与物不同,人可以自省,如果不能察己之善恶,那么就是恶了。

\section{第六讲、自立吾理:程颢哲学的精神(下)}
\subsection{四、	生之谓性}
程颢说“生之谓性,性即气,气即性,生之谓也”。“人生气禀,理有善恶”。人之生是禀气而生,当然有善恶。“继之者善”,所有的万物都继此生理,就是善。所有继承天地生生不已之道德都是善的。真正的人性应该从“成之者性也”的角度讲。具体万物的本性都有变化,这时候善恶才有了分别。
程颢充分肯定了人作为普通人存在的合理性,普通人的欲望。上天创造所有东西,我们首先要给一个肯定的态度。儒家认为人都是有限的,如实地去面对人的本质,所以不是说水浑浊了用清水去替换,而是就着这个浑水去改变。
“天命之谓性,率性之谓道,修道之谓教”。万物都有各自的本性,万物按照各自的本性来就是道,但是只有人才能“修道”,才有教化。所以恶的本质不是恶,二是过或不及。
通过“生之谓性”,他带出了几个道理:第一,强调人和万物的本性都是继天地生生之道而来的,所以人与万物的自我保存都是在保存天地生生之道;第二,所有的事物,从生理角度讲都是善的,所谓的恶是善的行为的过或不及,人才有善恶。修了过或不及,人的善恶问题就解决了。
程颢没有预设一个复杂的东西,不是“强生事”的。邵雍就有一点强生事。
\subsection{五、	以觉言仁}
程颢的“仁”有三个方面:以一体言仁,以生意言仁,以觉言仁。以觉言仁是核心,以生意言仁和以觉言仁是同一个思想的不同侧面。为什么仁者幸福,因为在真实处境中的醒觉的基础上我们才能真实地感受自己的存在。这里强调的是真实的处境和感受。
程颢有几大贡献。“生之谓性”回归易简,回归朴素,不是在现象背后去看到所谓的“真实”,现象本身就是真实,就是“诚”的体现,都继承了天地生生之道。另一大发明就是拈出了“敬”字,为儒学在修身的方法论上找到了真正的精神核心。
“以觉言仁”是“以一体言仁”的基础。为什么人能感受到和天地万物为一体,因为你能真正地觉知到天地万物的存在。“义礼智信皆仁也”。个体的消亡是生生不已的环节。明白这个仁的道理,用诚敬存之就可以了,不需要提防、穷思。
\subsection{六、	定性}
什么叫“定性”?就是心不扰动。“圣人之喜,以物之当喜,圣人之怒,以物之当怒”。但是圣人的喜怒不是主观的,而是“系于物也”。圣人应该也有应物,但是这样就是“累”,就会有心的扰动。程颢说,“犹累于外物”是因为你犹有内外之别,有“自我”的限制。如果没有限制,你会发现所有的天地万物都是与你有通感关系的。程颢也讲“定”,他说“知止而自定”。知道自己的分限就自然能定。一方面,我们对天地万物都有爱,另一方面,我们是在具体的社会生活中面对、实践这个爱。我们有自己的分位,能安于自己的分位,自然就不受扰动。

\section{第七讲、气本与神化:张载的哲学构建(上)}
\subsection{一、	虚与气}
张载认为,实存的世界是由两种存在形态构成的:太虚与气。太虚不是不存在,“太虚无形,气之本体”。这里“体”是“本来样子”的意思。“其聚其散,变化之客形尔”。天地万物的存在,不过是气的不同形态。
这里有几个要点。第一,太虚与气是并存的。第二,太虚也是气的一种状态,不是“无”。“知太虚即气,则无无”,太虚就是气的本来状态,是无形的状态,聚起来就是有形的状态。这分别从两个方面针对佛老。
一是佛老的宇宙论。“虚无穷,气有限”,是说太虚是没有任何具体限定的,气是有限定的。“体用殊绝”,体用被隔断了,道家这样的思想显然是错误的。佛家为什么说空?因为世界万物都是虚假的,所以空,背后的太虚才是真实的。张载认为气的不同状态都是真实的。
第二是虚气指向一种生死观。批判道教的“徇生执有”,留恋有限的形体的。批判佛教生死观的寂灭。
张载的生死观是,形体是有消散的,我的真常本性却永远存在。二程则批判张载的生生不已是相对的而不是绝对的。那么二程的生死观呢?它们认为,个体的消亡是生生不已实现的逻辑循环。如果个体可以不消亡,那么生生不已就是可以停止的。人消亡了,就不反了。“死之事即生是也”,告诉我们要过好每一天的生活。
张载实际上构造了一个气本论的传统。在虚气转化的时候注意到,虚和气不是整齐划一的,它们是并存的两种状态,没有有无,只有幽明。

\section{第八讲、气本与神化:张载的哲学构建(下)}
\subsection{二、	形与象}
有形一定有象,有象不一定有形,形比象第一个层次。有象而无形的阶段,如太虚。形和象怎么区分呢?形很清楚,就是有形体的,更多地属阴,而纯阴纯阳之物是没有形体的。有形的是属阴的,是被动的,而象这个层面是积极的。象是通过可感的、可直接看见的、物的某种形态的变化,体现出物物之间关系的变动,而且难以用单一感官把握整体的变化和变动。象一定要通过有形之物体现出来,否则就体会不到。
太虚是有象无形的,气和万物都是有象有形的。张载认为,有象无形的太虚已经是形而上的,所以张载那里的形上形下就是无形有形的区别。
\subsection{三、	参两}
参两就是叁两。首先来看一和两的关系。张载说天道为神,地道为物,天道地道相区别。神是最高的存在,连象都没有。“有两亦一在,无两亦一在,若无两焉用一”。张载认为没有对立的两体存在的、没有分化的世界是不存在的。而且只有两没有一世界也是分裂的。所以,“一物两体,气也;一故神,两故化,此天之所以参也”。分化的万物之中有统一的神贯穿其中,所以“本一故神”。两体就是太虚与气,始终贯通的就是“一”,正因如此,虚实、动静、聚散、清浊才能相互感通、转化。
除此之外,张载还说“化”。首先,神和化是体和用的关系,神是完全无法把捉的,化是能点点滴滴感知的,就是两体物物之间的相互感通的变化关系。
张载也说“变”。张载认为,世界是一个连续的“化”的过程,化到一定程度就“变”。化是永远的、不显著的、不可察觉的变化的积累;到了变就是化积累到了显著的阶段,可以察觉了。
张载说“神为不测,化为难知”。神和化结合,神是化背后的推力,化是神的推动带来的两体之间的相互作用的结果。
回顾一下结构:虚与气——形与象——叁两。“叁两”谈地道与天道,地道为两,天道为叁。叁就是“一物两体”。
\subsection{四、	感}
为什么张载要强调差异的普遍存在,这是为了安顿“感”。如果没有差异的存在,那么何来的“感”呢?所以说差异是感的逻辑环节,如果相同就无所谓感了。两体是根本差异,但是万物也有具体差异。
张载有三种感。第一种是“天地阴阳二端之感”,就是两体之感,这种感都是正感。另一种感是“人与物蕞然之感”,“蕞”就是小。这就是人与物之间比较狭隘的感,到此就有诚与妄的区别,有真实的部分,也有伪妄的部分。所以人生才迷茫,我们才有各种情与伪出来,被各种东西所影响。第三种是圣人之感。圣人是真正能通天地万物之情的,实际上是“人与物蕞然之感”向“天地阴阳二端之感”的回归。
为什么要强调感呢?还是为了破佛教。佛教强调断根尘,归空门,如果证明了感的普遍性,也就证明了人与万物普遍的关联,从而证明了儒家强调的伦常关系的合理性。
\subsection{五、	人性论}
张载区分了天地之性与气质之性。“天地之性”是纯善无恶的,“气质之性”是有善有恶的。“气质之性”在张载这儿体现为几个方面。张载讲“气”讲的是厚薄、清浊,但这只是构成气的材料,还要从结构上说。人和物的区别就是在于禀得气质之厚薄清浊的不同。由贯通虚与气的作用来说,这个贯通的作用是性。为什么是性?因为感是性的神妙的作用,性是感的内在动力和结构,因此感发动出来就是性。性通乎性之外,命行乎性之内。所以说,气质方面有不可改变的,比如“死生修夭”,即寿命长短。“感者性之神,性者感之体”,这是说性是人向外关联感通的倾向,如果没有“蕞然之感”的遮蔽,没有气质的遮蔽,就应该是无所不感的。但是人有气质,气质有清浊,所以就有所区别。
但是一直感下去就可能流于兼爱。爱是普遍的,感通也是普遍的,但是在具体的实施上是有差别的。禀气清的人可以感受很远,浊的人只能感受到身边或自己。修养就是使气由浊反清。
\subsection{六、	性与心}
性不能去除,没有主动性,但心有。人之所有可以修养,最重要在于发挥心的作用。那么怎么发挥?这就要修养工夫。
修养工夫的第一步就是要变化气质,即“虚心”的过程:外面是变化的气质,里面体现为虚心。什么是虚心?就是没有自己主观的成见。儒家强调的是内外交养,一方面通过身体的变化改变自己的内心,另一方面通过内心的变化来改变身体。通过气质变化,我们就有了进一步向道理开放自己的可能,以一种平和的公正的态度看待自己和他人。接下来就要去面对道理了。“大其心”其实就是“穷理”,就是要研究天下万物的道理。张载讲两类知识,“闻见之知”和“德性所知”,要以后者来统领前者。闻见之知就是通过物与物相感得到的知识。德性所知强调的是道德行为的出发点,道德一定是源自自己内心的价值取向的,而不是源自对象自身的经验品质 。它源自我们内心本性中一种固有的倾向,这种倾向是我们所要觉知的最根本的东西。我们首先要觉知出自己对他人的那种感通的关系,觉知到我们内心中的天地之性,然后我们意识到自己的气质之性对我们天地之性的遮蔽,然后通过穷理,一点一点扩充出去到极致。通过“大其心”,对“德性之知”有了真切的了解,然后用“德性之知”去驾驭和引领“闻见之知”,讲自己对天地万物的体贴落到实处。

\section{第九讲、形上定体:程颐的思想(上)}
\subsection{一、	形上形下}
程颢的形上形下太过圆融。程颐说“一阴一阳之谓道,道非阴阳也”。阴阳到了器这个层面,就有分别了。又阴又阳就不是一,是二了,就有分别,就不是形而上的。有了阴阳,就有刚柔、始终、聚散、消长等。程颐认为理、道或者说形而上者,是一阴一阳的所以然,即背后的根据。形上者不是空洞的“无”,它只是不可见,但有内在的一定的道理;这个道理和万物的所以然,和万物的秩序是一致的;这不是人为的安排,而是自然的;由于不是人为的,所以这又是易简之道。形而上者无形无象,但其中已具万理。
\subsection{二、	体用一源}
“体用一源,显微无间”是对道器关系的最准确的描述。理是难以察觉的,象是最明白显著的,理为体,象为用。“体用一源”不是说体用相同、体用合一。
\subsection{三、	生生之理}
首先,程颐对张载那种循环论的、气本论的进行了批判。张载的气本论预设了固定不变的不可消灭的质料。二程不能接受不可消灭这个观念。程颐讲“道则自然生万物”,气本身就是相机不已的,是在绝对的虚无中纯粹地创造。
在程颐这里,构成万物的气就是一种无尽的绝对的永恒的创造。理是永恒的,生生之理是永恒的,而所有的气都是创造出来的,最终也会尽灭无余。
\subsection{四、	道无无对}
世间没有单独的事物,无独必有对。道无无对强调了下面几点。第一,强调了一种均衡的世界观。第二,强调了对立,即彼此之间感应的无处不在。第三,强调天人感应。

\section{第十讲 形上定体:程颐的思想(下)}
\subsection{五、	公与仁}
程颐认为“仁之道,要之只消道一公字”。如果说程颢是“以觉言仁”,那么程颐更多是“以公言仁”。当然,不能说“公”就是“仁”,“公只是仁之理”,“公”落实在人身上就体现为仁。以“公”字为核心,就可以体现“物我兼照”,既可以看到对象的特性,也可以深刻理解自己,从而深切理解他人,这样才能“恕”,才能“爱”。
\subsection{六、	人性论}
程颐的人性论强调“性”与“才”。程子说“论性,不论气,不备;论气,不论性,不明”。要通过性、气、习三个方面才能对人性有全面理解。
程颐说“性即理也”。“理”是万物的所以然,“性”是人的本质,也是“理”。这个“理”对所有人都是统一的。“才禀于气”,而气有清浊,所以才有贤有愚。“才”和“气质之性”不一致。“才”更多的是在说“气禀”所带来的人的一种潜在的可能性,这种潜能有其固定的方向。气禀或气质之性本身是有善恶的,而才确是无所谓善恶的。
在程颐这里“天地之性”是“理”的性,相当于张载说的天地之性。另一方面是气禀,从用的角度讲叫“才”,从清浊的角度讲叫“气”。程颐的人性论的基本架构就是“理的性”和“气的性”的分别。
\subsection{七、	主敬}
“敬”是什么呢?程颐说“敬是闲邪之道,闲邪则诚自存”,“闲”是防卫。人是这样一种状态:人是始终处在和他者的内在本质关联中的、有限的、有情感的、终有一死的、牵挂着的存在。程颐的意思是,不好的东西一旦被克服掉了,好的东西自然就能彰显。“邪”就是不正,就是过或者不及。这是真正意义上的易简之道。不正的东西清除了,正的东西自然就出来了。程颐说“所谓敬者,主一之谓敬”。“主一”就是某种精神凝聚专一的状态。“所谓一者,无适之谓一”。没有任何具体对象的情况下,内心保持收敛、整齐、不放纵、专一的状态。“主一”就是“无适”的状态,没有任何对象和方向的一种精神的凝聚、迥然在中的状态。
程颐不说静,静就是无事。儒家的修养工夫不是静坐,而是在事上磨炼。
\subsection{八、	格物致知}
程颐说“格犹穷也,物犹理也”。格物就是穷理,研究事物的道理。穷理可以从多方面着手,读书讲明义理;讨论古今人物,分辨是非;在应对事物的时候,对所有事物都恰当地对待。这之中已经包含了对客观事物的探索。这样一种格物观念,讲明了明道理。明理是根本。“格物致知”是第一步,下面还要有“诚意”,就是要解决知行不合一。

\section{第十一讲、理气动静:朱子的哲学(上)}
\subsection{一、	体用}
“天地万物之理,无独必有对”,“对”中最重要的是“体用”。朱子用水比喻体用时,说水的各种具体动态变化是用,而以水的物理特性为基础的各种可能性是体。以身体为喻,我们的身体是体,目视、耳听、手足运动就是用。从现成的事物的结构上讲,静态的结构是体,以此静态结构而发生的种种运用就是用。就阳来说,阳体阴用;就阴来说,阴体阳用。朱子这里的阴阳互为体用。总体来说,朱子讲体的时候,更多地从静去讲;讲用的时候,更多地从动去讲。但不绝对,比如上面说的阴阳。那么,朱子说的体用是什么呢?
涉及两个方面的问题。第一,是“始”和“终”的问题。在朱子看来,体用在朱子那里,有时体现为始终,这样时间出来了。第二是“动静”。有了动静,空间就有了。有体必有用,只要是在时空中,就已经是具体的运用了。朱子批判空洞的体。儒家的仁义礼智显然是“体”,是天理层面的,属于形而上的,它们有分别。也就是说,不是完全无形无象无内容的东西才是“体”。
总的来说,朱子认为,有体必有用。一般来说,静的为体,动的为用。体用的概念,具体体现为“始终”和“动静”。这样就有了时空,体就有了用。
\subsection{二、	太极}
“太极”讲的就是“理”的问题。在朱子那里,天理是阴阳的所以然,太极是万物的所以然。
“无极而太极”就是“无形而有理”。太极就是天理,而天理是无形的。那么这里“理”的具体内涵是什么呢?可以从“格物致知”入手。在朱子那里,格物就是穷理,可以看他格物的方向,来看他要穷什么理。《大学或问》讲的格物的范围,从人伦到自然无所不包。研究的目标是什么呢?通过这样的实践,看到所有事物中的“所当然而不容已”与“所以然而不可易”。“不容已”有两种解释:不得不这样或必须得这样。“不可易”是指固有的、恒常的根据。天理就是“所当然”,就是“应然”。
那么,天地万物是如何“应然”的?朱子也强调了“不得不如此做”的必然性。“春生了便秋杀”,有始有终,是万物的必然性。“天命之当然”是必然性与应然性的统一。我们为何要选择这样的生活方式?因为这样符合天地生生之道。在朱子看来,这个世界就是一个生生不息的存在整体,没有一个虚无的阶段。但是,“动静无端,阴阳无始”的动力来自哪里?是作为根本的“太极”和“天理”,这就是天地生生的必然和应然。
天理既是必然又是应然,那么何者为主?“所当然”是天理的根本。所当然就是“做你应该做的”。所有的“应然”都要弄清楚几个问题:第一,为什么必然要这么做?第二,为什么可以这么做?所有的“应然”里面都有“能然、当然、必然、自然”的完整展开。以“当然”为核心,“能然、必然、自然”的完整展开,就是“天理”的真正内涵。最高的天理就是生生之理。
天理是所当然的具体化。天理作为实体,是具体的,而非抽象空洞的。
\subsection{三、	理与气}
对于程朱理学,有几点必须明确:第一,最真实的、始终存在的是“理”而不是“气”;第二,凡是“气”,皆有消尽之时,但是理却不毁。
如此以来,就有很多问题。首先,既然理是结合着“必然性”的“具体化的所当然”,那么,理和气谁先谁后?再者,理如何有动静?再次,既然万物根源于天理,怎么会产生如此有差异的世界?
(一)首先来看理生气的问题。理如何能生气呢?“气者,理之用”,有体必要用。一旦落实到用,就展开为动静和始终、时间和空间。所谓理生气,就是体有用的意思。阴阳就像两面扇磨,磨来磨去生成许多渣滓就有了复杂万殊的世界。
气不是理之外的另一元,只是理之用。理之体必然会展现为用,即展现到气的层面,即生生不已。“理必有气”。
更具体地说,只要有这个“理”,这个理就会体现为“用”,就体现为时空中某种可感的东西,体现出形象。
既然理是生生不已的,那么每时每刻都有新的阴阳之气,这样就和旧的阴阳之气产生了冲突。只要有生理,就是有限的,就是有始终的,就要经历元亨利贞或者生长收藏四个阶段。一个阶段还没结束就有新的有限的事物生出来了,冲突便出来了。其实善恶就是过和不及。该结束的不结束,就是恶。
(二)接下来是理气先后的问题。一般来说,不能说谁先谁后,而说谁是根本。显然,理是第一性的。因为只要是气,就会完全地消灭。理气先后不是时间上的问题,而是逻辑上的问题。
(三)下面是理气动静。关于理气关系,一个关键问题是太极是怎么动的。如果太极有动静,那太极岂不是具体的物?太极是形而上的,自然就不能有动静。既然太极不在时空中,那太极怎么动呢?怎么可以说“太极动而生阳”呢?这有两种解释。一是“太极之理包含动静之理”。这是有问题的,不符合“一本”原则。朱子告诉我们的就是一个生生之理,所有的事物统一于一个生生之理。动静与阴阳的关系,不是动了才有阳,静了才有阴,动和阳是一个阶段,静和阴是另一个阶段。太极之动便是阳,太极之静便是阴。
天理或者太极作为天命之当然,也就是所当然的具体化,这意味着一种肯定的倾向。有了肯定,自然就有否定。而肯定的倾向就有了动的意思,否定的倾向就有了静的意思,不是太极另外包含了一个动静。所当然具体体现在就是“应当怎样”,包含了“应当这样”和“应当不这样”,分别有一个动和静的意思。这就是“太极有动静”的含义。“理有动静”和“理必有气”是完全一致的。对于具体事物而言,具体事物运动,事物中的理也在运动。关于理气动静,要分层面来看:一个是太极动而生阳的问题,一个是具体事物的理的动静问题。朱子说“太极者本然之妙,动静者所乘之机”。“妙”是创生义,“机”就是门轴,门轴中间是不动的,而门的运动围绕不动的中轴。这表明,动以静为条件,而门轴的静,为门的动创造了条件。
(四)最后来说“理一分殊”。理一分殊要解决这样的问题:万物形成以后,太极是否还寓于万物之中?若是,这个太极与作为生生不已的根源的太极有什么关系?这个问题的关键在于,每一个个别事物当中的太极和作为本体的太极之间的关系。所谓个别事物的太极,就是具体时空关系里那个恰当的点,或者恰当的分寸。所有的事物都体现为具体的时空当中的差异性,每一处差异性当中都有其极好的至善的分寸。那么万物有没有本性?有的,每个事物都禀得“天地生生之理”,都要经过元亨利贞的过程,有始有终。人的本性体现在哪里?体现在每个动作。
我们对理一分殊做了一个区分:一方面,太极是极好至善处,每种具体处境都有其极好至善处;另一方面,万物继天地生理而生,天地生生之理具体化到每个事物上,也就构成事物的本性。
太极不是一个具体的有形象的东西,而是始终作为一种动态的倾向作用在我们身上的。这种倾向根源于天地生生之理。就是根源于一个“仁”字。仁里面也就包含了义礼智信。

\section{第十二讲、理气动静:朱子的哲学(下)}
\subsection{四、	人性}
在人性上,朱子基本强调程子和张载。他认同要从“天命之性”和“气质之性”来把握人性。那么问题是,“天命之当然”怎么会生出如此驳杂的气质之性呢?其实,气质之性具体地讲,就是具体人物的复杂的刚柔变化。
对于人物理气同异的问题,朱子说“论万物之一源,则理同而气异;观万物之异体,气犹近而理绝不同”。万物的理是相同的,气禀不同;讨论万物差异的时候,理确是不同的。
\subsection{五、	心性情意}
首先来看性和情的关系。朱子说“性体请用”。“情”有几类:以仁为本性,义为本性,礼为本性,智为本性。此外还有七情。七情是从属于四端的。
朱子重视“心统性情”的说法。有两层意思:一是“心包性情”。心有性情两个方面,都是心的表现。二是“心主性情”。仁会发恻隐,但能否转化为行动就要心的主体性作用。性和情是相对,而心和性情是相对的。“所当然”发动处就是情,该恻隐就恻隐。从性到情是一个必然,心是主宰,情的发显和节制都是心决定的。
什么是“意”呢?“意则有主向”,意有了确定方向的情。意就是有了具体的、确定的对象的情。
此外,还有“志”。“志是心之所之”。“志”和“意都是属于情的”,区别在于,意是往来经营的。意是志的脚,志必须通过意才能实现。
\subsection{六、	涵养}
朱子继承了程子“涵养须用敬”的思想。“敬有甚事,只如畏字相似”。敬是比较接近畏的。畏没有具体的对象,是心灵的整齐收敛。关于中和问题,朱子认为心还是有一个思虑未萌的阶段,就是喜怒哀乐之未发。在这个阶段,心没有任何具体的思虑,没有任何具体的心灵内容。
朱子认识到“静处涵养此心”的重要性。
\subsection{七、	致知}
朱子也把“格”解释为“至”,这至少有三层含义。一是“即物”,也就是接触事物,尊重事物的客观性;二是“穷理”,研究事物的道理;三是“至极”,把研究的道理推向极处。即物穷理需要一个长期的积累过程,只能一点点积累。在朱子看来,研究的事物越多,格物越多,心就越明灵,人的认识能力就越强。

\section{第十三讲、自作主宰:陆九渊与朱陆之辩}
\subsection{一、	陆九渊的思想}
(一)本心。 陆九渊本心的概念更多的是一种不可遏制的道德情感,其实就是孟子说的“四端”。本心是每个人都具体的饱满的道德情感,他人遇到痛苦,你就有恻隐之心。本心就是人心的本来样子,人心未受遮蔽、污染之前,是纯善的。人们会有恶,是有人欲的遮蔽。
(二)心即是理。陆九渊的理是有客观性的,“此理乃宇宙之所固有”,所以这个理和朱子大概是一致的。那么本心和天理有什么关系呢?“心即理也”。如果本心和天理分开,那么就是“二本”了。一个有客观性普遍性必然性的理是万物的基础,人的本心是人行为的基础。所有人的心本质上都是相同的心,所有的万物之理本质上都是相同的理,最纯粹的道理并无二致。所以,此心此理不容分别。
(三)收拾身心,自作主宰。儒家强调自立、自主,这是自由的本质。只有这种自主精神、心灵最高的主动状态,才能真正地让道德情感发显,不受遮蔽。
(四)格物与静坐。“格物者,格此者也”,“此”就是“我”,是此心之理。所以在陆九渊看来,格物就是“正此心”。只要把遮蔽心灵的物欲清除掉,此理自然就会发显出来。陆九渊也强调静坐。
(五)义利之辨。“义利之辨”是儒家最根本的原则之一。但我们要防止两种倾向,第一只讲“利”,第二把“义”“利”绝对对立。陆九渊说“辨志”,辨明一个人的志向,想成为一个怎样的人。
\subsection{二、	朱陆之辩}
(两个人打嘴炮,略。)

\section{第十四讲、无善无恶:王阳明的心学}
\subsection{一、	心外无理}
阳明说“心外无理”。朱子认为要先知后行,但是知行之间有一个转渡者。阳明要强调的是:所有的道德行为都得发源于一种完善健全的道德人格。那么,具体知识重要吗?朱子认为,任何一个完善的过程都要经历“格物、致知、诚意、正心、修身”的过程,把这个过程视为一个完整的道德行为发生的过程,人的道德境界只有在具体的道德实践中才能提高。阳明不同,他强调的是一颗纯善的心。在他看来,我们应该把追求完善的道德人格作为自己的目标,如此所有的念头和行为就都是善的。这样一来,外在的客观知识就成了不必要的。但这不是抛弃外在的物理。“物理,吾心”本是一体,不断向外寻求,反而荒废了主体。在阳明看来,所有的道德行为都从一个主体的道德意志、道德意识出发,这种道德意志不是来自于道德对象,而是来自于道德主体。从“心”作为道德行为的发动者这个角度上讲,所有的善都不在心之外,善源自心灵的道德意志的发动。
另一个问题是,客观的物理是否在此心之外。在这里,阳明没有否认客观知识在道德实践中的必要性。讲求客观知识,首先要看你的目的是什么,是否出于一个为善的心。所有,必须要有一个引领者——完善的道德人格。为此,就必须“存天理去人欲”。阳明认为人的恶来源于物欲对本心的遮蔽,一旦除去人欲此心便纯是天理。在这里,阳明强调的是探求客观物理的主观条件。
“理也者,心之条理也”,“条理”就是心灵的本质倾向,或者心灵的结构。心灵发挥作用时,必然沿着这样的方向。在这个意义上,所有的善都不在心的条理之外。
\subsection{二、	心外无物}
“心外无物”不是说心外没有事物存在,而是说即使有物,也不是我们心中的物。他所讲的“心外无物”指的是在人类的行为之内所牵涉到的物。这个物是跟意识关联在一起的。正因为有人的灵明,才有今天意义上的物。“心外无物”的“心”可以理解为“客观精神”。比如看花,不用一片一片去辨认,整朵花在你心中完整地呈现,说明你心中有花的概念,一眼看去就知道是花。如果没有这个概念,是不可能明白的。从这个角度说,阳明的“理”是在人的行为之内、跟人的行为有关的。
\subsection{三、	格物}
在朱程那里,知和行之间有一个“诚意”功夫。阳明在解释格物的时候说“格者,正也。正其不正,以归于正也”。“其”是我们心中的意念。事物本身是没有道德本质、无所谓善恶的。善恶的分别源于道德主体的善恶。所以,只要“正其不正”就好了。
\subsection{四、	知行合一}
如果知行是合一的,那么为什么处处能见到知而不行的情况呢?阳明讲知行合一有几个方面:第一,他首先强调的是知行的本体。阳明当时讲知行合一针对的是知行分隔的状况。知行之所以不能合一,就是未能做到诚意。“知行合一”首先强调的是一个知行本体,因为他认为知行本来就应该是统一的。是什么导致不统一?是人欲的遮蔽。第二,“真知”的概念:真知就一定能行。但我们的知识一般不是真知,因为大多数都是口耳相传,没有在自己身心上验证过。第三,他还讲“知是行之始,行是知之成”。知行是一个连贯的完整过程。
阳明还认为动了一个不好的念头就已经是行了。
\subsection{五、	致良知}
“良”是本有、固有的意思。“良知”就是一个好恶之心,就是一个知是知非的心。知是知非的能力是人们完整地具备的,但这种知是知非的能力还不能变成具体的道德实践,因此还需要一个推扩的过程,“致良知”就是这样一个过程。在什么地方扩充呢?在“存天理灭人欲”上做功夫。从这一念之微出发,复得这良知完完全全,然后任此良知妙用,依次应对世间万事。
“无善无恶心之体,有善有恶意之动,知善知恶是良知,为善为恶是格物”。这里“体”是本来面目的意思。心的本来面目就是不执着于善恶,不管善恶,心总能自然、无滞。阳明有个比喻:心灵好像眼睛,往眼睛里撒沙子,眼镜睁不开,撒黄金碎屑,同样睁不开,眼镜里容不下任何东西,这与善恶无关。“有善有恶意之动”是说,你的心不可能总处在无善无恶的本然状态,心必然发为意,一旦发为意,就有善有恶了。“知善知恶是良知”,人一念发动就知道它是善是恶,之后就可以笃实地去为善为恶了。

\section{第十五讲、理只是气之理:罗钦顺的思想}
\subsection{一、	异学驳论}
罗钦顺真正要面对是儒学内部的异端思想,他认为其根源在于禅宗。罗钦顺主要批评王明阳。首先是在经典的解释上。阳明将“格”解释为“正”,“物”又是意之所在,格物不就是正心吗。既然可以正心,意又是心之所发,不就自然做到了诚意?既然如此,《大学》何必搞出格物、致知、诚意、正心来。另一个质疑是《朱子晚年定论》。阳明认为朱子晚年已经颇有悔悟,强调“向内用功”了。罗钦顺批评阳明对资料随意增改。
在罗钦顺看来,心学的根本问题在于不自觉中陷入了禅宗。在他看来,儒与禅的根本分别在于“圣人本天,佛氏本心”。他认为不应该把人的觉和知作为天地的根底,而要把天地作为人的根底。我们要把自己的身心放到天地万物一体中看,我们继天地生生之理而生,所以要承担照料万物的责任。
罗钦顺对两宋道学传统都所有继承,可以看作道学在明代的发展。对于整个心学,罗钦顺始终在批判。
\subsection{二、	理只是气之理}
罗钦顺强调“理只是气之理”。朱程认为“所以一阴一阳之谓道”,有了“所以”,才有形上形下的区分,才能极大拓展宋明理学的哲学思维的空间。罗钦顺恰恰认为这是个问题。他认为,有了“所以”,理和气的分别就被过分地强调了。在朱子看来,“气只是理之气”,“理必有气”。而罗钦顺强调的是理只是气之理。朱子认为天理是没有主动创造的功能的,所以“气,强于理”,在具体的万物的运行上理是管不了气的。罗钦顺质疑:如果认为“气强于理”,那么所谓太极又怎么能是“造化之枢纽,品物之根底”呢?罗钦顺这里说的“作为气的理”是什么呢?有两种可能,一是作为气的属性的理,如气的刚柔、动静、清浊;二是理是气运行的基本规律。这二者是不同的。他说“理须就气上认取”,理一定要在气上才能看到,“然认气为理不行”,既不能认理为气,也不能分别开来。“理只是气之理,当于气之转折处观之”,“转折处”就是消长转化之际。跟张载一样,罗钦顺也认为气是不会灭尽无余的,气的往来是有这样一个理的主宰的。在罗钦顺的思想架构中,首先,作为质料的气是永恒的,不会消灭的;其次,气是有往来的。而往来根源于感应的普遍性,感应的过程有其既定的模式,就是理。“理只是气之理”,只是感应的固定渠道或模式。
“理,一也,必因感而后形”,“感则两也,不有两则无一”,感与应必然是两个事物之间的关系,一个事物不能感。所以说,罗钦顺说的“理只是气之理”,更准确的表达是理只是感应之理。
罗钦顺强调理的必然性、客观性与自然。所谓的自然就是消除了人的主观。由于天地万物运动不息,但它们的感应关系是固定有条理的,所以理又是有确定性的。这样一来,也就是自然不可违的。从此,罗钦顺开辟出了一条对人的合理欲望充分宽容的路径。
总结一下,罗钦顺的基本哲学架构是张载式的,在他看来,理其实就是感应之条理,有客观性、必然性、自然性和确定性,所以人就应该按照这种条理来生活。
\subsection{三、	论性}
罗钦顺认为不能说心即理,只能说性即理。那么心的作用是什么呢?人的本性就是道体本身,因此是无为的,没有主观的造作。罗钦顺认为知觉的能力在于人心。心是有动静的,理则是静的,性是静态的体。罗钦顺追求心与理一,就是心能够充分了解自己所有的性和理。
\subsection{四、	格物观}
罗钦顺认为,天地间无一物不是我们的分内事,反对向外,就是分别了内外,就有了边界。天地间所有人物都根源于一气之感应,罗钦顺的理是往而来,来而往的感应之理,没有一个作为实体的生生之理作为创造的根源。天地之间,只有气化的往来感应,是永不停息的,所以是造化之枢纽。既然我们与万物都有普遍感应,也就对万物有普遍责任,万物也就不在我们之外。在他看来,格物就是要建立起心物之间的通彻无间的关联,通过格物我们看到事物感应的条理,我们心的感应条理也因此慢慢呈现出来。


\end{document}
